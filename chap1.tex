\chapter{Introduction}
\label{chap:Introduction}

%% Restart the numbering to make sure that this is definitely page #1!
\pagenumbering{arabic}

The field of neutrino physics is currently undergoing a revolution.  With its tenuous postulation~\cite{PauliOpenLetter} acting as a future omen, the neutrino's mark on history would not become apparent from its discovery~\cite{Cowan20071956, PhysRevLett.9.36, Kodama2001218}, but rather from a spate of surprising discoveries at the end of the 20th century~\cite{PhysRevLett.81.1562, PhysRevLett.87.071301, PhysRevLett.90.021802} which conclusively proved that the Standard Model, while very successful, was incomplete.  This revelation was experimental proof of Maki, Nagakawa and Sakata's extension~\cite{Maki01111962} to Pontecorvo's theory of neutrino oscillation ~\cite{Pontecorvo} with the inclusion of the Mikheyev-Smirnov-Wolfenstein (MSW) effect~\cite{PhysRevD.17.2369,Mikheev:1986gs}.  The findings were groundbreaking as the underlying theory requires massive neutrinos, which is in direct contradiction to the Standard Model.  The, now, standard theory of neutrino oscillation defines three neutrino flavours and three neutrino masses.  However, the map between flavour and mass is not one-to-one, but rather a rotation of mass space onto flavour space as
\begin{equation}
\begin{pmatrix}
\nu_e \\
    \nu_\mu \\
    \nu_\tau
\end{pmatrix}
\end{equation}

\section{The state of the field}
\label{sec:StateOfTheField}
Neutrino oscillations are really kool
