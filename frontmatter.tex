%% Title
\titlepage[Imperial College London] {
  A thesis submitted to Imperial College London\\ for the degree of Doctor of Philosophy}

%% Abstract
\begin{abstract}%[\smaller \thetitle\\ \vspace*{1cm} \smaller {\theauthor}]
  %\thispagestyle{empty}
  This thesis presents an analysis of neutrino interactions using the Electromagnetic Calorimeters (ECals) of the Tokai-to-Kamioka (T2K) off-axis near detector (ND280) as a target using data collected during T2K run 3C.
  \newline
  \newline
  The analysis presented shows the development of a new set of reconstruction algorithms which are able to reconstruct multiple tracks which originate from the same neutrino interaction.  The output of this reconstruction is the used as the basis of a $\nu_\mu$ charged current inclusive selection in the ND280 ECals.  The selected events are then used in a simple $\chi^2$ fit to extract the T2K flux-averaged $\nu_\mu$ charged current inclusive cross-section on lead, which is measured as $\langle \sigma^{\textrm{CC}}_{\textrm{Pb}} \rangle_{\phi} = 8.13^{+1.33}_{-1.26} \times 10^{-39} \textrm{ cm}^2 \textrm{ nucleon}^{-1}$.
  \end{abstract}


%% Declaration
\begin{declaration}
The copyright of this thesis rests with the author and is made available under a Creative Commons Attribution Non-Commercial No Derivatives licence. Researchers are free to copy, distribute or transmit the thesis on the condition that they attribute it, that they do not use it for commercial purposes and that they do not alter, transform or build upon it. For any reuse or redistribution, researchers must make clear to others the licence terms of this work.
\newline
\newline
The work presented in this thesis was made possible due to a large number of collaborators involved with and beyond the T2K experiment.  Citations have been used where possible to reference work not done by the author.  When this is not possible, the people responsible have been explicitly named.
\newline
\newline
The main contributions by the author to this thesis have been the development of a new set of reconstruction algorithms for the ECal which allow neutrino interactions to be studied, development of the $\nu_\mu$ charged current inclusive selection in the ECal, evaluation of the ECal detector systematics and implementation of the $\chi^2$ fit (the original idea for this fit was by M. Scott).
\newline
\newline
The author has also made many contributions to the T2K experiment which have not been discussed in this thesis.  These contributions largely involved package management of the ECal reconstruction algorithms (ecalRecon) and the final analysis file producing software (oaAnalysis).
  \vspace*{1cm}
  \begin{flushright}
    Dominic Brailsford
  \end{flushright}
\end{declaration}


%% Acknowledgements
\begin{acknowledgements}
I would like to thank Dr. Asher Kaboth and Dr. Mark Scott for their enormous help during my time on the T2K experiment.  You two have provided excellent guidance and patientlly answered a great many questions during my time as a student and this analysis would not have progressed without your help.
\newline
\newline
I would also like to thank my supervisor, Dr. Yoshi Uchida.  Your gentle nudges have aimed me in the correct direction numerous times and I can not help but think that this thesis would be a mess without your supervision.
\newline
\newline
Thanks also go to Dr. Morgan Wascko.  The many times I had to quickly come to your office to discuss an analysis problem have been such a huge help to me.
\newline
\newline
I would also like to thank Dr Per. Jonsson for the advice and many thought provoking discussions during evaluation of the ECal systematics.
\newline
\newline
I also want to give many thanks to the coffee break group, Dr. Ben Smith, Dr. Peter Sinclair, Pip Hamilton, Luke Pickering and Wing Ma.  Discussion of my analysis during this time has been a huge help.
\newline
\newline
Kelly, thank you so much for putting up with me, even more so when writing my thesis.  I know I can't have been the easiest person to deal with a lot of the time, but you remaining by my side has kept me sane.
\newline
\newline
Finally, the largest thanks go to my parents.  I can never repay you for all you have done for me.  No matter how far I go or what I acheive in life, it is because of both of you.
\end{acknowledgements}


% Preface
%\begin{preface}
%  \noindent
%  This thesis describes my analysis of the $\nu_\mu$ charged-current cross-section on lead using the 
%  T2K near detector electromagnetic calorimeters.
%
%\end{preface}

%% ToC
\tableofcontents


%% Strictly optional!
\frontquote{%
  These chickens jackin' my style.
}%
  {Fergie}
%% I don't want a page number on the following blank page either.
\thispagestyle{empty}
